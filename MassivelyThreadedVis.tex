% -*- latex -*-

%%%%%%%%%%%%%%%%%%%%%%%%%%%%%%%%%%%%%%%%%%%%%%%%%%%%%%%%%%%%%%%%%%%%%%%%%%%%%
% This metadata is specific for the sig-alternate document class.

\documentclass{sig-alternate}

\conferenceinfo{Ultravis}{2013 Denver, CO USA}
\CopyrightYear{2013} % Allows default copyright year (20XX) to be over-ridden - IF NEED BE.
%\crdata{0-12345-67-8/90/01}  % Allows default copyright data (0-89791-88-6/97/05) to be over-ridden - IF NEED BE.

\title{A Classification of Scientific Visualization Algorithms for Massive Threading}

% ACM has this complicated setup for authors that only really works for 6
% or less and is pretty big and pretty ugly in any case. This is amore
% compact representation.

\numberofauthors{1}

\author{
  \alignauthor
  Kenneth~Moreland,$^{\ddagger}$ Berk~Geveci,$^*$ Kwan-Liu~Ma,$^{\dagger}$ ...\\
  \affaddr{$^{\ddagger}$Sandia National Laboratories}\\
  \affaddr{$^*$Kitware, Inc.}\\
  \affaddr{$^{\dagger}$University of California at Davis}\\
}

% End of metadata specific for sig-alternate document class.
%%%%%%%%%%%%%%%%%%%%%%%%%%%%%%%%%%%%%%%%%%%%%%%%%%%%%%%%%%%%%%%%%%%%%%%%%%%%%

\usepackage{amsfonts}
\usepackage{amssymb}
\usepackage{amsmath}
\usepackage{graphicx}
\usepackage{varioref}
\usepackage{fancyvrb}
\usepackage{ifthen}
\usepackage{cite}
\usepackage{subfig}
\usepackage{xspace}
\usepackage[pdfborder={0 0 0}]{hyperref}
\usepackage{verbatim}

\usepackage{color}
\definecolor{yellow}{rgb}{1,1,0}
\definecolor{black}{rgb}{0,0,0}
\definecolor{ltcyan}{rgb}{.75,1,1}
\definecolor{red}{rgb}{1,0,0}
\definecolor{gray}{rgb}{.6,.6,.6}
\definecolor{darkred}{rgb}{0.5,0,0}
\definecolor{darkgreen}{rgb}{0,0.5,0}

% Cite commands I use to abstract away the different ways to reference an
% entry in the bibliography (superscripts, numbers, dates, or author
% abbreviations).  \scite is a short cite that is used immediately after
% when the authors are mentioned.  \lcite is a full citation that is used
% anywhere.  Both should be used right next to the text being cited without
% any spacing.
\newcommand*{\lcite}[1]{~\cite{#1}}
\newcommand*{\scite}[1]{~\cite{#1}}

\newcommand{\etal}{et al.}

\newcommand*{\keyterm}[1]{\emph{#1}}

\newcommand{\fix}[1]{{\color{red}\textsc{[#1]}}}

% Avoid putting figures on their own page.
\renewcommand{\textfraction}{0.05}
\renewcommand{\topfraction}{0.95}
\renewcommand{\bottomfraction}{0.95}

% Make sure this is big enough so that only big figures end up on their own
% page but small enough so that if a figure does have to be on its own
% page, it won't push everything to the bottom because it's not big enough
% to have its own page.
\renewcommand{\floatpagefraction}{.75}

\newenvironment{packed_itemize}{
\begin{itemize}
  \setlength{\topsep}{0pt}
  \setlength{\itemsep}{0pt}
  \setlength{\parskip}{0pt}
  \setlength{\parsep}{0pt}
  \setlength{\partopsep}{0pt}
}{\end{itemize}}

\hyphenation{Para-View Map-Re-duce}

\begin{document}

\maketitle

\begin{abstract}

As the number of cores in processors increase and accelerator architectures
are becoming more common, an ever greater number of threads is required to
achieve full processor utilization. Our current parallel scientific
visualization codes rely on a partitioning of data to achieve parallel
processing, but this approach will not scale as we approach massive
threading in which work is distributed in such a fine level that each
thread is responsible for a minute portion of data. In this paper we
characterize the challenges of refactoring our current visualization
algorithms by considering the finest portion of work each performs and
examining the domain of input data, overlaps of output domains, and
interdependencies among work instances. \fix{Some further sentence about
  number of classes and how it will help focus research.}

\end{abstract}

\section{Introduction}

\noindent
Processor clock and execution rates have flatlined. Instead, successive
generations of processors provide more parallel threading
capability\lcite{Sutter2005}. Recent CPUs feature multiple cores and
hyperthreading technology to allow each core to run concurrent
threads. Furthermore, accelerator type architectures, which have
lightweight cores grouped to share control, are becoming increasingly
popular for their high price and power to performance ratios.

High performance computing is also seeing a remarkable increase in the
parallelism required on large-scale systems. Consider, for example, the
last two generations of leadership-class computers at the Oak Ridge
Leadership Computing Facility. The previous Jaguar-XT5 system had a peak
performance of about 2 petaflops using about 200 thousand concurrent
processes. The current Titan-XK7 system, which incoporates GPU
accelerators, has a peak performance of over 20 petaflops but can require
70 to 500 \emph{million} concurrent threads in order to achieve that.

Production visualization products today achieve parallel scalability using
a data parallel method that relies on partitioning the data into
independent domains for each process\lcite{Ahrens2001}. Each domain is
processed independently, so ghost regions and overlap are required at
domain boundaries so that the interaction of work on parallel processes can
be ignored. However, this approach is infeasible when dealing with massive
amounts of threads on these emerging architectures. We now need to design
new algorithms with a key on data interdependencies to process efficiently
in these massively multithreaded environment.

\bibliographystyle{abbrv}
\bibliography{MassivelyThreadedVis}

\end{document}
