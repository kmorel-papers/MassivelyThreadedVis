% -*- latex -*-

%%%%%%%%%%%%%%%%%%%%%%%%%%%%%%%%%%%%%%%%%%%%%%%%%%%%%%%%%%%%%%%%%%%%%%%%%%%%%
% This metadata is specific for the sig-alternate document class.

\documentclass{sig-alternate}

\conferenceinfo{Ultravis}{2013 Denver, CO USA}
\CopyrightYear{2013} % Allows default copyright year (20XX) to be over-ridden - IF NEED BE.
%\crdata{0-12345-67-8/90/01}  % Allows default copyright data (0-89791-88-6/97/05) to be over-ridden - IF NEED BE.

\title{A Classification of Scientific Visualization Algorithms for Massive Threading}

% ACM has this complicated setup for authors that only really works for 6
% or less and is pretty big and pretty ugly in any case. This is amore
% compact representation.

\numberofauthors{1}

\author{
  \alignauthor
  Kenneth~Moreland,$^{\ddagger}$ Berk~Geveci,$^*$ Kwan-Liu~Ma,$^{\dagger}$ ...\\
  \affaddr{$^{\ddagger}$Sandia National Laboratories}\\
  \affaddr{$^*$Kitware, Inc.}\\
  \affaddr{$^{\dagger}$University of California at Davis}\\
}

% End of metadata specific for sig-alternate document class.
%%%%%%%%%%%%%%%%%%%%%%%%%%%%%%%%%%%%%%%%%%%%%%%%%%%%%%%%%%%%%%%%%%%%%%%%%%%%%

\usepackage{amsfonts}
\usepackage{amssymb}
\usepackage{amsmath}
\usepackage{enumitem}
\usepackage{graphicx}
\usepackage{varioref}
\usepackage{fancyvrb}
\usepackage{ifthen}
\usepackage{cite}
\usepackage{subfig}
\usepackage{xspace}
\usepackage[pdfborder={0 0 0}]{hyperref}
\usepackage{verbatim}

\usepackage{color}
\definecolor{yellow}{rgb}{1,1,0}
\definecolor{black}{rgb}{0,0,0}
\definecolor{ltcyan}{rgb}{.75,1,1}
\definecolor{red}{rgb}{1,0,0}
\definecolor{gray}{rgb}{.6,.6,.6}
\definecolor{darkred}{rgb}{0.5,0,0}
\definecolor{darkgreen}{rgb}{0,0.5,0}

% Cite commands I use to abstract away the different ways to reference an
% entry in the bibliography (superscripts, numbers, dates, or author
% abbreviations).  \scite is a short cite that is used immediately after
% when the authors are mentioned.  \lcite is a full citation that is used
% anywhere.  Both should be used right next to the text being cited without
% any spacing.
\newcommand*{\lcite}[1]{~\cite{#1}}
\newcommand*{\scite}[1]{~\cite{#1}}

\newcommand{\etal}{et al.}

\newcommand*{\keyterm}[1]{\emph{#1}}

\newcommand{\fix}[1]{{\color{red}\textsc{[#1]}}}

% Avoid putting figures on their own page.
\renewcommand{\textfraction}{0.05}
\renewcommand{\topfraction}{0.95}
\renewcommand{\bottomfraction}{0.95}

% Make sure this is big enough so that only big figures end up on their own
% page but small enough so that if a figure does have to be on its own
% page, it won't push everything to the bottom because it's not big enough
% to have its own page.
\renewcommand{\floatpagefraction}{.75}

\newenvironment{packed_itemize}{
  \begin{itemize}[noitemsep]
}{
  \end{itemize}
}

\newcommand{\algclass}[1]{\textsf{#1}}
\newcommand{\alg}[1]{#1}

\newcommand{\algorithmclasssection}[1]{
  \vspace{\baselineskip}\noindent{\large\textbf{\algclass{#1}}}}

\newcommand{\algorithmclass}[5]{
  \algorithmclasssection{#1} %
  \vspace{-.4\baselineskip}
  \begin{description}[leftmargin=9em,style=nextline,noitemsep]
    \raggedright
  \item[Input] #2
  \item[Output] #3
  \item[Interdependence] #4
  \item[Algorithms] #5
  \end{description}
}

\hyphenation{Para-View Map-Re-duce}

\begin{document}

\sloppy

\maketitle

\begin{abstract}

As the number of cores in processors increase and accelerator architectures
are becoming more common, an ever greater number of threads is required to
achieve full processor utilization. Our current parallel scientific
visualization codes rely on a partitioning of data to achieve parallel
processing, but this approach will not scale as we approach massive
threading in which work is distributed in such a fine level that each
thread is responsible for a minute portion of data. In this paper we
characterize the challenges of refactoring our current visualization
algorithms by considering the finest portion of work each performs and
examining the domain of input data, overlaps of output domains, and
interdependencies among work instances. \fix{Some further sentence about
  number of classes and how it will help focus research.}

\end{abstract}

\section{Introduction}

\noindent
Processor clock and execution rates have flatlined. Instead, successive
generations of processors provide more parallel threading
capability\lcite{Sutter2005}. Recent CPUs feature multiple cores and
hyperthreading technology to allow each core to run concurrent
threads. Furthermore, accelerator type architectures, which have
lightweight cores grouped to share control, are becoming increasingly
popular for their high price and power to performance ratios.

High performance computing is also seeing a remarkable increase in the
parallelism required on large-scale systems. Consider, for example, the
last two generations of leadership-class computers at the Oak Ridge
Leadership Computing Facility. The previous Jaguar-XT5 system had a peak
performance of about 2 petaflops using about 200 thousand concurrent
processes. The current Titan-XK7 system, which incoporates GPU
accelerators, has a peak performance of over 20 petaflops but can require
70 to 500 \emph{million} concurrent threads in order to achieve
that. Building algorithms that are implemented for new process
architectures and programming models and that support the massive
parallelization they require is considered one of the top research
challenges for scientific visualization\lcite{Childs2013}.

Production visualization products today achieve parallel scalability using
a data parallel method that relies on partitioning the data into
independent domains for each process\lcite{Ahrens2001}. Each domain is
processed independently, so ghost regions and overlap are required at
domain boundaries so that the interaction of work on parallel processes can
be ignored. However, this approach is infeasible when dealing with massive
amounts of threads on these emerging architectures. We now need to design
new algorithms with a key on data interdependencies to process efficiently
in these massively multithreaded environment.

Unfortunately, to perform scientific visualization with massive threading,
we need to redesign our algorithms to work effectively with fine-grained,
independent operations. There has already been significant work in making
scientific visualization algorithms work well with shared memory threading
and accelerator types of architectures\fix{cite,cite,cite}, but all these
projects focus on the implementation of a single or select fixed set of
algorithms. Our goal in this paper is to present commonalities among
various algorithms that share parallel-programming challenges. By
addressing this higher level challenges, we can make better progress to
ensure that our scalable scientific visualization needs are met.

To identify these high level parallel-programming challenges, we first
revisit the key principles on which we base our current parallel
visualization algorithms (Section~\ref{sec:KeyPrinciples}) and then
categorize our current set of scientific visualization algorithms based on
behavior of the fundamental computation
(Section~\ref{sec:Classification}).


\section{Key Principles}
\label{sec:KeyPrinciples}

\noindent
Our current scalable scientific visualization relies on a coarse
partitioning of data into domains that can often be processed
independently. Law \etal\scite{Law1999} provides the following three key
principles that must be satisfied for this independent processing to behave
correctly.

\begin{description}
\item[Data Separability] The data can be broken into domains in a way that
  is simple and efficient. Furthermore, algorithms behave properly on
  independent domains.
\item[Mappable Input] Given an identification for a portion of the output,
  the input domain responsible for this output can be determined. The
  output portion can, and often is, identified as simple piece $i$ of $N$.
\item[Result Invariant] The output of the algorithm is equivalent across
  all possible partitioning.
\end{description}

At first glance, it would appear that any algorithm abiding these key
principles could be partitioned indefinitely. However, there are two
problems that arise when building a massively threaded algorithm in which
the data is partitioned (potentially) down to domains of single elements.

The first problem is that many algorithms do not really have a clear
mapping from an algorithm's output to its input. For example, when
computing a contour\lcite{Lorensen1987}, it is seldom practical to know a
priori how large the output will be, from what domain of the input each
piece of the output will originate, and what the distribution of elements
in the output partitions will be. As long as the data partitioning is
coarse enough, managing uneven or even empty domains is
inconsequential. Load imbalance in downstream processing can either be
resolved by dynamic rebalancing or, more commonly, simply tolerating
it. However, to achieve efficency with massive threading, it is important
to know the precise elements on which to schedule the working
threads. Thus, it is often more practical to determine partitioning by
mapping from input to output rather than output to input.

The second problem is that although plenty of algorithms are result
invariant in that different partitioning results are \emph{equivalent}, the
structures they build are not strictly \emph{isomorphic}. Partitioning data
often results in duplicate information on domain boundaries. \fix{Add
  figure and description demonstrating duplicated points when slice is
  partitioned.}

This duplication of results is a convenient mechanism to avoid
communication in parallel processing, and in coarse-level parallelism the
duplication is easily managed through ghost regions. However, if massive
threading reduces each domain to every independent element, the duplication
is dramatic and creating ghost regions is infeasible. It is therefore not
practical to assume result invariance for most algorithms. Instead we have
to identify and characterize the \keyterm{interdependence} of the work
where concurrent threads produce coincident data or the threads otherwise
require collective operations. It is only then that we can design
strategies to manage the work interdependence.

With these issues in mind, we provide an analogous set of key principles
for the operation of scientific visualization algorithms for massive
threading.

\begin{description}
\item[Data Separability] The data can be broken down to an elemental level
  fine enough to provide independent work for sufficient threads.
\item[Discoverable Input Mapping] The existance of output elements can be
  efficiently determined from the input.
\item[Collective Work] In the cases where work is not independent, the
  overlap of responsibility can be resolved through efficient collective
  operations.
\end{description}


\section{Classification of Visualization Algorithms}
\label{sec:Classification}

\noindent
To better understand the effort involved with updating our scientific
visualization algorithms to massively threaded processors, we look into the
behavior of the algorithms we are currently using in production
tools. Specifically, we consider those algorithms available in ParaView, a
popular open-source scalable scientific visualization
application\lcite{ParaView}.

For each algorithm in ParaView, we consider the smallest unit of data on
which the algorithm can be sensibly run. Given this unit of
\keyterm{input}, we identify what \keyterm{output} is created for
it. Assuming that many threads are simultaneously producing output, we also
identify the \keyterm{interdependence} among these threads.

From this information we derive classes of scientific visualization
algorithms, where the algorithms of each class share the same input,
output, and interdependence characteristics. This classification is
constructive because once we resolve the challenges of mapping input to
output and managing interdependence with collective operations, the
algorithms within each class are straightforward to implement from their
serial counterparts.

\algorithmclass{Basic Mapping}
               {Field element (single)} % Input
               {Field element (single)} % Output
               {None} % Interdependence
               {Append Attributes, Append Datasets/Geometry, Calculator,
                 Elevation, Generate Ids, Process Ids, Reflect, Transform,
                 Warp}

\noindent
The \algclass{Basic Mapping} algorithms operate on an array of field values
and generate another array of field values. Each output field value is
computed independently using only the associated input value (or values if
there is more than one input array).

The typical purpose of this type of algorithm is to generate a derived
field of data. Some of these operate on the coordinates of mesh points, for
example the \alg{Transform} filter moves and warps the data with an affine
transformation. However, when point coodinates are treated as a field, the
behavior characteristics are the same.

The \algclass{Basic Mapping} class is essentially just a parallel
\texttt{for} operation over input and output field arrays. As such, basic
mapping is essentally built in to many multi- and many-core languages and
APIs including OpenMP (parallel for pragma\lcite{Quinn2004}), CUDA (triple
chevron notation\lcite{Sanders2011}), Thrust (for\_each generic
algorithm\lcite{Thrust}), and Intel Threading Building Blocks
(parallel\_for generic algorithm\lcite{TBB}). Although all of these systems
use significantly different syntax, Baker \etal\scite{Baker2010} show how
to simplify porting using a generic programming interface over them.


\algorithmclass{Map by Cell}
               {Cell topology (single)} % Input
               {Field element (single)} % Output
               {None} % Interdependence
               {Cell Centers, Point Data to Cell Data, Cell Derivitives,
                 Mesh Quality}

\noindent
The \algclass{Map by Cell} class is very similar to \algclass{Basic
  Mapping} with the exception that the input field arrays are not a
one-to-one mapping to the output field arrays. Instead, these operations
are performed using the information over a complete cell and attached
fields.

Operations of this class involve characterizing the shape of the cell (as
in the case of \alg{Cell Centers} and \alg{Mesh Quality}) or math
operations that operation on a continuous function rather than discrete
values (as is the case for \alg{Cell Derivitives}).

Although parallel threads generally do access overlapped portions of the
input data, the calculations computed by each thread are independent and
there is no overlap in the output they produce and are thus easy to
parallelize. The EAVL library and Dax toolkit each provide a generic
algorithm to perform \algclass{Map by Cell}\lcite{EAVL,Moreland2011:LDAV}.


\algorithmclass{Reconnect Cell}
               {Cell topology (single)} % Input
               {Cell connections (zero or more)} % Output
               {None} % Interdependence
               {Extract Cells by Region, Extract Selection, Threshold,
                 Triangulate}

\noindent
There are several apparent patterns in algorithms that build a topological
structure, the first of which is \algclass{Reconnect Cell}. The
characteristics of these algorithms are that the output topology uses the
same points as (or a subset of the points from) the input topology and that
each output cell depends on exactly one input cell. In essence the
connectivity of each cell is being redefined.

Because each thread creates an independent list of cell connections,
\algclass{Reconnect Cell} algorithms have no interdependence. This makes
them very similar to \algclass{Map by Cell} algorithms with the important
exception that most \algclass{Reconnect Cell} algorithms have a variable
number of outputs across the threads that is not known at the onset of
execution. Thus, these algorithms must implement some form of
\keyterm{stream compaction} to build efficient packed arrays for the
output. Also, since some of these algorithms only use a subset of the input
points, it may be desirable to explicitly identify and extract these
points.

Each of three SDAV many-core frameworks for visualization\lcite{Sewell2012}
provides an example of \alg{Threshold}, a \algclass{Reconnect Cell}
algorithm, using a different approach. PISTON uses a stream compaction
algorithm to determine an efficient output layout and then generates the
output cells in parallel\lcite{PISTON}. New points are created in the
output, which implicitly removes points from the input at the expense of
replicating points in the output. Dax uses a similar stream compaction but
outputs a more compact array of connection
identifiers\lcite{Maynard2013}. Dax can also optionally extract the subset
of points used in the output at an added computational expense. EAVL
provides a different approach wherein the output data structure references
the input structure and the data management makes this behave as an
independent data structure\lcite{Meredith2012}. This approach is much
faster and memory efficient, but there is no explicit representation of the
result for use in other packages that might expect that and it is not
possible to remove the input structure from memory as long as the output
still references it.


\algorithmclass{Build Independent Topology}
               {Cell topology (single)} % Input
               {Cell topology (constant number)} % Output
               {None} % Interdependence
               {Glyph, Shrink, Tube}

\noindent
Algorithms that \algclass{Build Independent Topology} create new geometry
requiring both new points and new cells that connect these points. The
geometry created by each thread in this algorithm class is completely
independent from that created in any other thread; there are no topological
connections between them.

For example, consider the \alg{Glyph} filter. Each thread in this filter
produces a small 3D object, like a scaled sphere or oriented arrow,
centered at the point assigned to the thread. Each 3D object is completely
disconnected (topologically) from its bretheren and thus can be created
independently and concurrently.

The behavioral characteristics of the \algclass{Build Independent Topology}
algorithms are very similar to the \algclass{Basic Mapping} and
\algclass{Map by Cell} algorithms. The only meaningful difference between
them is the interpretation of the output data. Instead of producing a
single or set of field values, \algclass{Build Independent Topology}
algorithms produce topologies of cells, vertices, and point coordinates;
these data must be interpreted and managed as such.


\algorithmclass{Build Connected Topology}
               {Cell topology (single)} % Input
               {Cell topology (zero or more)} % Output
               {Duplicate constituent elements of output} % Interdependence
               {Clean, Clip, Contour, Extrusion, Isovolume, Slice, Subdivision}

\noindent
The most technically complicated class of algorithms we find that generate
geometry are those that \algclass{Build Connected Topology}. These
algorithms create new geometry requiring both new points and new cells
that connect these points. Of the points created by each thread many or all
of them may be coincident with points created by another thread. These
coincident points represent connections across elements created by
different threads; capturing these connections creates an interdependence
across threads.

The easiest solution to this problem is to simply ignore it by letting
coincident points exist and losing the topological connections. Thus, the
output of the algorithm is topologically a set of disconnected cells, often
referred to as a \keyterm{soup}. We find this approach common, particularly
in implementations of \alg{Contour} for GPUs\fix{cite,cite,cite}.

Producing a soup of cells may be an acceptable solution in some situations,
particularly if the result is to be rendered and then forgotten, but can
also be a problematic solution. The first problem is that this redundancy
causes an inflation of the memory required. The second problem is that
subsequent algorithms, such as estimating smooth normals on a contour,
might require the lost topological connections. Existing tools,
particularly those built on VTK\lcite{VTK}, expect chains of algorithms of
this nature to work.

A typical approach to finding coincident points in a serial algorithm is to
use an iterative \keyterm{locator} structure to find any coincident points
that are already created\lcite{VTKUsersGuide}. However, this approach is
unlikely to work in a massively threaded environment as constant updates to
the locator will require far too much synchronization. Bell\scite{Bell2010}
informally proposes a \keyterm{vertex welding} algorithm that can
efficently find these coincident points on massive threading after the
fact, but a solution working more closely to the algorithm might have
better results.


\algorithmclass{Capture Cell Adjacencies}
               {Constituent element (single) with incident cells (varies)} % Input
               {Field or cell connections for constituent element (single)} % Output
               {Duplicate constituent elements of input} % Interdependence
               {Curvature, Cell Data to Point Data, Extract Edges, Extract
                 Surface (external faces), Feature Edges, Gradient
                 Estimation, Normal Generation}

\noindent
The previously listed algorithms all operate on a single point in the mesh
or on a single cell and its incident features. These data are quickly
locatable in general data structures. However, some algorithms need to
\algclass{Capture Cell Adjacencies}. They typically operate by considering
the points, edges, or faces in the mesh and examining the cells incident to
them.

In some types of data, particularly structured data with implicit
topologies, enumerating the mesh elements and finding incident cells is
trivial. However, in unstructured data represented by cell connection
lists, this information is not explicitly stored. Thus, although the
computation of \algclass{Capture Cell Adjacencies} algorithms are
completely independent, they might require a collective operation to
identify the domain of the input each thread needs.

Serial VTK algorithms support cell adjacencies by building a
\keyterm{links} array that lists which cells each point is incident on. A
similar array could be used in a massively threaded environment, but since
a links array is not always available we first need an efficient massively
threaded algorithm to build one.

An alternate approach is to use a different cell representation that does
explicitly capture these incidence relationships. Such alternate
representations include half-edge structures\lcite{Kettner1998}, cellular
data structures\lcite{Alumbaugh2005}, and circular incident edge
lists\lcite{Levy2001}. The drawback to these linked representations is that
storing links to every desired incidence relationship is costly. Also, most
existing data representations, including CGNS\lcite{CGNS}, VTK\lcite{VTK},
and XDMF to name a few, use cell connection lists, so a links array would
likely need to be built to convert between the representations anyway.


\algorithmclass{Globally Reduce}
               {Varies} % Input
               {Aggregated value (single)} % Output
               {Global reduction on intermediate values} % Interdependence
               {Histogram, Integrate, Outline, Statistics}

\noindent
Description.


\algorithmclass{Probe Data}
               {Searchable structure, Point location (single)} % Input
               {Varies} % Output
               {None} % Interdependence
               {Probe Location, Quadric Clustring, Resample with Dataset,
                 Streamlines, Streaklines}

\noindent
Description.



\algorithmclasssection{Remaining Algorithms}

Extract Subset (from structured data)

Image based algorithms: kernel convolutions (gradient) and FFT.

D3

Connectivity
Delaunay


\bibliographystyle{abbrv}
\bibliography{MassivelyThreadedVis}

\end{document}
